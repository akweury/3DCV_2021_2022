\documentclass[a4paper, twoside, english]{article}

\usepackage{amsmath}
\usepackage{amsfonts}
\usepackage{ihci}
\usepackage{graphicx}
\usepackage{subfig}

\graphicspath{{./../figures/}}
\newcommand{\br}{\textbf{R}}
\newcommand{\qed}{\hfill \ensuremath{\Box}}


\title{Exercise 3 \\ 3D Computer Vision}  % Replace "Template Report" with Exercise 1, Exercise 2, etc
\author{Jingyuan Sha}                       % Replace with your names
\date{\today}                              % Replace with current date

\begin{document}

\maketitle


\section{Theory}

\begin{enumerate}
	\item let $ r_1, r_2$ be the first column of $ \textbf{R}_0 $ and second column of $ \textbf{R}_0 $, $ \textbf{x}_0 = (X,Y)$, then $ \textbf{x}_1 $ can be written as follows
	\begin{equation*}
		\textbf{x}_1 = \textbf{K}[r_1r_2\textbf{t}_0]\left(\begin{matrix}
			X\\
			Y\\
			1
		\end{matrix}\right)
	\end{equation*}
	\item \begin{enumerate}
		\item Since $ \textbf{K}, \textbf{R}, \textbf{t} $ are known, thus we can calculate fundamental matrix $ F $.
		\item Based on $ \textbf{F},  \textbf{x}_i $ , we can get epipolar lines $ \textbf{l}_i $ for both images.
		\item Determine the corresponding ray of $ \textbf{x}_i $ in each image by epipolar lines and extrinsic matrix. 
		\item The 3D point $ X $ is determined by the intersection of rays.
	\end{enumerate}
	\item Epipole can be computed as follows:
	\begin{equation*}
		\textbf{e}_0 = \textbf{P}_0\textbf{C}_1, \ 
		\textbf{e}_1 = \textbf{P}_1\textbf{C}_0
	\end{equation*}
	The relation between epipolar line and epipole is
	\begin{equation*}
		\textbf{l}_1 = \textbf{F}_0\textbf{x}_0, \ 
		\textbf{l}_0 = \textbf{F}_1\textbf{x}_1,
	\end{equation*}
	\item We can use 8-Point algorithm to compute $ \textbf{F} $ given $ n\geq 8 $ corresponding points $ \textbf{x}_i $ and $ \textbf{x}_i' i=1,...,n$, then the fundamental matrix $ F $ is the solution of equation system
	\begin{equation}
		\textbf{x}_i^{'T}\textbf{F}\textbf{x}_i = \textbf{0},\  1\leq i \leq n
	\end{equation}
	\item The fundamental matrix can be computed with intrinsic parameter and poses under equation
	\begin{equation*}
		F = K^{'-T}[t]_{\times}RK^{-1}
	\end{equation*}
	\item A fundamental matching problem of matching technique is aperture problem. It is make sense to match features in this way when the object moves align with the direction of indicated direction.
	
\end{enumerate}


\newpage


\end{document}