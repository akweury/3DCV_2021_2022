\documentclass[a4paper, twoside, english]{article}

\usepackage{amsmath}
\usepackage{amsfonts}
\usepackage{ihci}
\usepackage{graphicx}
\usepackage{subfig}

\graphicspath{{./../figures/}}
\newcommand{\br}{\textbf{R}}
\newcommand{\qed}{\hfill \ensuremath{\Box}}


\title{Exercise 2 \\ 3D Computer Vision}  % Replace "Template Report" with Exercise 1, Exercise 2, etc
\author{Jingyuan Sha}                       % Replace with your names
\date{\today}                              % Replace with current date

\begin{document}

\maketitle

\part{Theory}

\section{Homography Definition}

\begin{enumerate}
	\item  A homography is a non-singular, line preserving projective mapping $ h: P^n \mapsto P^n $. (from slide page 9.)
	\item the degree of freedom is $ (n+1)^2-1 $
\end{enumerate}

\section{Line Preservation}
	Let mapping $ h(\textbf{x}_i) = \textbf{H}\textbf{x}_i $, where $ \textbf{x}_i $ lies on line \textbf{l} such as $ \textbf{l}\textbf{x}_i=0 $. Then we have 
	\[ \textbf{l} \textbf{H}^{-1}\textbf{H}\textbf{x}_i=0 \]
	Then the mapping point $ \textbf{H}\textbf{x}_i $ lies on line $ \textbf{l}\textbf{H}^{-1} $. Thus the mapping points in projective space preserves lines.

\section{Camera Center in World Coordinates}

\begin{enumerate}
	\item The camera matrix 
	\begin{equation*}
		\begin{bmatrix}
			R & t \\
			0 & 1\\
		\end{bmatrix}
	\end{equation*}
	can map a homogeneous world coordinate to camera coordinate. Let $ R_C $ be the rotation matrix with respect to the world coordinate axes. Then, camera matrix is equal to the inverse of world coordinate matrix, that is
	\begin{equation*}
	\begin{split}
	\begin{bmatrix}
		R & t \\
		0 & 1\\
	\end{bmatrix}
	& = \begin{bmatrix}
		R_C & C_W \\
		0 & 1\\
	\end{bmatrix}^{-1}\\
	&= \left(
	\begin{bmatrix}
		I & C_W \\
		0 & 1\\
	\end{bmatrix}
	\begin{bmatrix}
		R_C & 0 \\
		0 & 1\\
	\end{bmatrix}\right)^{-1}\\
	& = \left(
	\begin{bmatrix}
		R_C & 0 \\
		0 & 1\\
	\end{bmatrix}\right)^{-1} 
	\left(
	\begin{bmatrix}
		I & C_W \\
		0 & 1\\
	\end{bmatrix}\right)^{-1}\\
	& = 
	\begin{bmatrix}
		R_C^T & 0 \\
		0 & 1\\
	\end{bmatrix}
	\begin{bmatrix}
		I & -C_W \\
		0 & 1\\
	\end{bmatrix}\\
	& = \begin{bmatrix}
		R_C^T & -R_C^TC_W \\
		0 & 1\\
	\end{bmatrix}
	\end{split}
	\end{equation*}	
	Then we have 
	\begin{equation*}
		\begin{split}
			R &= R_C^T \\
			t &= -R_C^TC_W
		\end{split}
	\end{equation*}
	That is $ C_W = -R_C\cdot t = -R^T \cdot t $

	\item the vector $ \textbf{t} $ points the position of world origin in camera coordinates.
\end{enumerate}
\newpage

\part{Implementation}



\end{document}