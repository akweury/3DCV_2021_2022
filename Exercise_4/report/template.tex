\documentclass[a4paper, twoside, english]{article}

\usepackage{amsmath}
\usepackage{amsfonts}
\usepackage{ihci}
\usepackage{graphicx}
\usepackage{subfig}
\graphicspath{{./../figures/}}
\newcommand{\br}{\textbf{R}}
\newcommand{\qed}{\hfill \ensuremath{\Box}}


\title{Exercise 4 \\ 3D Computer Vision}  % Replace "Template Report" with Exercise 1, Exercise 2, etc
\author{Jingyuan Sha}                       % Replace with your names
\date{\today}                              % Replace with current date

\begin{document}

\maketitle


\section{Theory}
% find correspondences
% if the epipolor geometry is known
% correspondence search reduces to a 1-D search
\begin{enumerate}
	\item the epipolar line of each point in a pair of rectified images is just along the horizontal scanlines. Thus in rectified images, with knowing a point, the correspondence can be searched along the scanline.
	\item The triangulation can be simplified based on rectified images, the Z-coordinate is depend on the disparity. Based on Z-coordinate, focal length, and image point, we can calculate X and Y coordinates.
	\item Image rectification is not a good approach in multi-view dense reconstruction, since the correspondences might not exist in any two images at the same time, which could lead to solving a similar triangle problem with infinity scale. 
\end{enumerate}

% image rectification simplify the search for correspondences


\section{Practical}

\begin{enumerate}
	\item 
	\begin{enumerate}
		\item see code
		\item the pixel based disparity map is shown in Figure 3.
			\begin{figure}[h!]
			\begin{minipage}[b]{0.3\textwidth}
				\includegraphics[width=\textwidth]{../data/medieval_port/left.jpg}
				\caption{Left.}
			\end{minipage}
			\hfill
			\begin{minipage}[b]{0.3\textwidth}
				\includegraphics[width=\textwidth]{../data/medieval_port/right.jpg}
				\caption{Right.}
			\end{minipage}
			\hfill
			\begin{minipage}[b]{0.3\textwidth}
				\includegraphics[width=\textwidth]{../pixelDisparity.jpg}
				\caption{pixel-based matching.}
			\end{minipage}
			\hfill
		\end{figure}
		\item pixel based disparity map is not a good representation of the original scene. Because there are too many possible matches on the other images, many points look similar.
	\end{enumerate}
	\item 
	\begin{enumerate}
		\item see code.
		\item The disparity maps with 7x7 window computed by ncc and ssd are shown in Figure 4 and 5. 
		\begin{figure}[h!]
			\begin{minipage}[b]{0.45\textwidth}
				\includegraphics[width=\textwidth]{../nccDisparity7x7.jpg}
				\caption{nccDisparity7x7.jpg}
			\end{minipage}
			\hfill
			\begin{minipage}[b]{0.45\textwidth}
				\includegraphics[width=\textwidth]{../ssdDisparity7x7.jpg}
				\caption{ssdDisparity7x7.jpg}
			\end{minipage}
			\hfill
		\end{figure}
		\item See figure 6-11
		\begin{figure}[h!]
			\begin{minipage}[b]{0.3\textwidth}
				\includegraphics[width=\textwidth]{../ssdDisparity3x3.jpg}
				\caption{ssd disparity 3x3.}
			\end{minipage}
			\hfill
			\begin{minipage}[b]{0.3\textwidth}
				\includegraphics[width=\textwidth]{../ssdDisparity5x5.jpg}
				\caption{ssd disparity 5x5.}
			\end{minipage}
			\hfill
			\begin{minipage}[b]{0.3\textwidth}
				\includegraphics[width=\textwidth]{../ssdDisparity7x7.jpg}
				\caption{ssd disparity 7x7.}
			\end{minipage}
			\hfill
			\begin{minipage}[b]{0.3\textwidth}
				\includegraphics[width=\textwidth]{../nccDisparity3x3.jpg}
				\caption{ncc disparity 3x3.}
			\end{minipage}
			\hfill
			\begin{minipage}[b]{0.3\textwidth}
				\includegraphics[width=\textwidth]{../nccDisparity5x5.jpg}
				\caption{ncc disparity 5x5.}
			\end{minipage}
			\hfill
			\begin{minipage}[b]{0.3\textwidth}
				\includegraphics[width=\textwidth]{../nccDisparity7x7.jpg}
				\caption{ncc disparity 7x7.}
			\end{minipage}
			\hfill
		\end{figure} 
		\item the quality of the disparity map improved a little bit in window-based results. Since the similar points are reduced based on a bigger matching window.
		\item see code
		\item See Figure 12.
	\end{enumerate}
	
	\item 
	\begin{enumerate}
		\item The result is shown in the figures.
		\item The result is shown in the figures.
		\item with bigger window size, the reconstruction quality goes better with fewer isolated areas, with smaller window size, the quality goes worse but with more details. With the size of window increasing, the the processing time increases as well.
	\end{enumerate} 
\end{enumerate}



\newpage


\end{document}