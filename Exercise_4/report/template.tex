\documentclass[a4paper, twoside, english]{article}

\usepackage{amsmath}
\usepackage{amsfonts}
\usepackage{ihci}
\usepackage{graphicx}
\usepackage{subfig}

\graphicspath{{./../figures/}}
\newcommand{\br}{\textbf{R}}
\newcommand{\qed}{\hfill \ensuremath{\Box}}


\title{Exercise 4 \\ 3D Computer Vision}  % Replace "Template Report" with Exercise 1, Exercise 2, etc
\author{Jingyuan Sha}                       % Replace with your names
\date{\today}                              % Replace with current date

\begin{document}

\maketitle


\section{Theory}
% find correspondences
% if the epipolor geometry is known
% correspondence search reduces to a 1-D search
\begin{enumerate}
	\item the epipolar line of each point in a pair of rectified images is just along the horizontal scanlines. Thus in rectified images, with knowing a point, the correspondence can be searched along the scanline.
	\item The triangulation can be simplified based on rectified images, the Z-coordinate is depend on the disparity. Based on Z-coordinate, focal length, and image point, we can calculate X and Y coordinates.
	\item Image rectification is not a good approach in multi-view dense reconstruction, since the correspondences might not exist in any two images at the same time, which could lead to solving a similar triangle problem with infinity scale. 
\end{enumerate}


% image rectification simplify the search for correspondences



\newpage


\end{document}